\documentclass{article}

\usepackage[margin=0.75in]{geometry}
\usepackage{amsmath,amssymb}
\usepackage{graphicx}
\usepackage{url}
\usepackage{hyperref}
\usepackage{float}

\begin{document}


\title{%
Pycaret Documentation in your own words \\
}

\author{
  Shokrzad, Reza \\
  \texttt{Group ?}
  \and
   Shokrzad, Reza \\
  \texttt{Group ?}
  \and
   Shokrzad, Reza \\
  \texttt{Group ?}
}
\date{February 2022}
\maketitle


\section{Summary}
%As briefly as you can, just write down main characteristics and purposes (for what it is useful?).


\section{Comparisons}
%Compare SKlearn to pycaret in some sentences. 



\section{Strengths}
%What are its Strengths in general? How this package makes you flexible?

\section{Weaknesses}
%what are the drawbacks in general?

\section{Problem Definition}
%Define a task that you are going to work on for your favorite dataset. After running your pipeline with both SKlearn and Pycaret add up some more details.

\section{Quality of the writing}
%Show your results in a proper form as tables or diagrams.

\section{Queries for discussion}
% Try to mention at least three logical questions about this package or all Python packages to discuss more.


\end{document}